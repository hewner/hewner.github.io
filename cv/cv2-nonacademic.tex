
% NOTE:
% This requires the package texlive-latex-extra

%%%%%%%%%%%%%%%%%%%%%%%%%%%%%%%%%%%%%%%%%%%%%%%%%%%%%%%%%%%%%%%%%%%%%%%%
%%%%%%%%%%%%%%%%%%%%%% Simple LaTeX CV Template %%%%%%%%%%%%%%%%%%%%%%%%
%%%%%%%%%%%%%%%%%%%%%%%%%%%%%%%%%%%%%%%%%%%%%%%%%%%%%%%%%%%%%%%%%%%%%%%%

%%%%%%%%%%%%%%%%%%%%%%%%%%%%%%%%%%%%%%%%%%%%%%%%%%%%%%%%%%%%%%%%%%%%%%%%
%% NOTE: If you find that it says                                     %%
%%                                                                    %%
%%                           1 of ??                                  %%
%%                                                                    %%
%% at the bottom of your first page, this means that the AUX file     %%
%% was not available when you ran LaTeX on this source. Simply RERUN  %%
%% LaTeX to get the ``??'' replaced with the number of the last page  %%
%% of the document. The AUX file will be generated on the first run   %%
%% of LaTeX and used on the second run to fill in all of the          %%
%% references.                                                        %%
%%%%%%%%%%%%%%%%%%%%%%%%%%%%%%%%%%%%%%%%%%%%%%%%%%%%%%%%%%%%%%%%%%%%%%%%

%%%%%%%%%%%%%%%%%%%%%%%%%%%% Document Setup %%%%%%%%%%%%%%%%%%%%%%%%%%%%

% Don't like 10pt? Try 11pt or 12pt
\documentclass[10pt]{article}

% This is a helpful package that puts math inside length specifications
\usepackage{calc}


% Simpler bibsection for CV sections
% (thanks to natbib for inspiration)
\makeatletter
\newlength{\bibhang}
\setlength{\bibhang}{1em}
\newlength{\bibsep}
 {\@listi \global\bibsep\itemsep \global\advance\bibsep by\parsep}
\newenvironment{bibsection}%
        {\vspace{-\baselineskip}\begin{list}{}{%
       \setlength{\leftmargin}{\bibhang}%
       \setlength{\itemindent}{-\leftmargin}%
       \setlength{\itemsep}{\bibsep}%
       \setlength{\parsep}{\z@}%
        \setlength{\partopsep}{0pt}%
        \setlength{\topsep}{0pt}}}
        {\end{list}\vspace{-.6\baselineskip}}
\makeatother

% Layout: Puts the section titles on left side of page
\reversemarginpar

%
%         PAPER SIZE, PAGE NUMBER, AND DOCUMENT LAYOUT NOTES:
%
% The next \usepackage line changes the layout for CV style section
% headings as marginal notes. It also sets up the paper size as either
% letter or A4. By default, letter was used. If A4 paper is desired,
% comment out the letterpaper lines and uncomment the a4paper lines.
%
% As you can see, the margin widths and section title widths can be
% easily adjusted.
%
% ALSO: Notice that the includefoot option can be commented OUT in order
% to put the PAGE NUMBER *IN* the bottom margin. This will make the
% effective text area larger.
%
% IF YOU WISH TO REMOVE THE ``of LASTPAGE'' next to each page number,
% see the note about the +LP and -LP lines below. Comment out the +LP
% and uncomment the -LP.
%
% IF YOU WISH TO REMOVE PAGE NUMBERS, be sure that the includefoot line
% is uncommented and ALSO uncomment the \pagestyle{empty} a few lines
% below.
%

%% Use these lines for letter-sized paper
\usepackage[paper=letterpaper,
            %includefoot, % Uncomment to put page number above margin
            marginparwidth=1.2in,     % Length of section titles
            marginparsep=.05in,       % Space between titles and text
            margin=1in,               % 1 inch margins
            includemp]{geometry}

%% Use these lines for A4-sized paper
%\usepackage[paper=a4paper,
%            %includefoot, % Uncomment to put page number above margin
%            marginparwidth=30.5mm,    % Length of section titles
%            marginparsep=1.5mm,       % Space between titles and text
%            margin=25mm,              % 25mm margins
%            includemp]{geometry}

%% More layout: Get rid of indenting throughout entire document
\setlength{\parindent}{0in}

%% This gives us fun enumeration environments. compactitem will be nice.
\usepackage{paralist}

%% Reference the last page in the page number
%
% NOTE: comment the +LP line and uncomment the -LP line to have page
%       numbers without the ``of ##'' last page reference)
%
% NOTE: uncomment the \pagestyle{empty} line to get rid of all page
%       numbers (make sure includefoot is commented out above)
%
\usepackage{fancyhdr,lastpage}
\pagestyle{fancy}
%\pagestyle{empty}      % Uncomment this to get rid of page numbers
\fancyhf{}\renewcommand{\headrulewidth}{0pt}
\fancyfootoffset{\marginparsep+\marginparwidth}
\newlength{\footpageshift}
\setlength{\footpageshift}
          {0.5\textwidth+0.5\marginparsep+0.5\marginparwidth-2in}
\lfoot{\hspace{\footpageshift}%
       \parbox{4in}{\, \hfill %
                    \arabic{page} of \protect\pageref*{LastPage} % +LP
%                    \arabic{page}                               % -LP
                    \hfill \,}}

% Finally, give us PDF bookmarks
\usepackage{color,hyperref}
\definecolor{darkblue}{rgb}{0.0,0.0,0.3}
\hypersetup{colorlinks,breaklinks,
            linkcolor=darkblue,urlcolor=darkblue,
            anchorcolor=darkblue,citecolor=darkblue}

%%%%%%%%%%%%%%%%%%%%%%%% End Document Setup %%%%%%%%%%%%%%%%%%%%%%%%%%%%


%%%%%%%%%%%%%%%%%%%%%%%%%%% Helper Commands %%%%%%%%%%%%%%%%%%%%%%%%%%%%

% The title (name) with a horizontal rule under it
%
% Usage: \makeheading{name}
%
% Place at top of document. It should be the first thing.
\newcommand{\makeheading}[1]%
        {\hspace*{-\marginparsep minus \marginparwidth}%
         \begin{minipage}[t]{\textwidth+\marginparwidth+\marginparsep}%
                {\large \bfseries #1}\\[-0.15\baselineskip]%
                 \rule{\columnwidth}{1pt}%
         \end{minipage}}

% The section headings
%
% Usage: \section{section name}
%
% Follow this section IMMEDIATELY with the first line of the section
% text. Do not put whitespace in between. That is, do this:
%
%       \section{My Information}
%       Here is my information.
%
% and NOT this:
%
%       \section{My Information}
%
%       Here is my information.
%
% Otherwise the top of the section header will not line up with the top
% of the section. Of course, using a single comment character (%) on
% empty lines allows for the function of the first example with the
% readability of the second example.
\renewcommand{\section}[2]%
        {\pagebreak[3]\vspace{1.3\baselineskip}%
         \phantomsection\addcontentsline{toc}{section}{#1}%
         \hspace{0in}%
         \marginpar{
         \raggedright \scshape #1}#2}

% An itemize-style list with lots of space between items
\newenvironment{outerlist}[1][\enskip\textbullet]%
        {\begin{itemize}[#1]}{\end{itemize}%
         \vspace{-.6\baselineskip}}

% An environment IDENTICAL to outerlist that has better pre-list spacing
% when used as the first thing in a \section
\newenvironment{lonelist}[1][\enskip\textbullet]%
        {\vspace{-\baselineskip}\begin{list}{#1}{%
        \setlength{\partopsep}{0pt}%
        \setlength{\topsep}{0pt}}}
        {\end{list}\vspace{-.6\baselineskip}}

% An itemize-style list with little space between items
\newenvironment{innerlist}[1][\enskip\textbullet]%
        {\begin{compactitem}[#1]}{\end{compactitem}}

% An environment IDENTICAL to innerlist that has better pre-list spacing
% when used as the first thing in a \section
\newenvironment{loneinnerlist}[1][\enskip\textbullet]%
        {\vspace{-\baselineskip}\begin{compactitem}[#1]}
        {\end{compactitem}\vspace{-.6\baselineskip}}

% To add some paragraph space between lines.
% This also tells LaTeX to preferably break a page on one of these gaps
% if there is a needed pagebreak nearby.
\newcommand{\blankline}{\quad\pagebreak[2]}

% Uses hyperref to link DOI
\newcommand\doilink[1]{\href{http://dx.doi.org/#1}{#1}}
\newcommand\doi[1]{doi:\doilink{#1}}

% For \url{SOME_URL}, links SOME_URL to the url SOME_URL
\providecommand*\url[1]{\href{#1}{#1}}
% Same as above, but pretty-prints SOME_URL in teletype fixed-width font
\renewcommand*\url[1]{\href{#1}{\texttt{#1}}}

% For \email{ADDRESS}, links ADDRESS to the url mailto:ADDRESS
\providecommand*\email[1]{\href{mailto:#1}{#1}}
% Same as above, but pretty-prints ADDRESS in teletype fixed-width font
%\renewcommand*\email[1]{\href{mailto:#1}{\texttt{#1}}}

\newcommand{\respub}[4]{
  \item #2.  \emph{#1.}  #3.  #4.
}

%%%%%%%%%%%%%%%%%%%%%%%% End Helper Commands %%%%%%%%%%%%%%%%%%%%%%%%%%%

%%%%%%%%%%%%%%%%%%%%%%%%% Begin CV Document %%%%%%%%%%%%%%%%%%%%%%%%%%%%

\begin{document}
\makeheading{Michael Hewner}

\section{Contact Information}
%
% NOTE: Mind where the & separators and \\ breaks are in the following
%       table.
%
% ALSO: \rcollength is the width of the right column of the table
%       (adjust it to your liking; default is 1.85in).
%
\newlength{\rcollength}\setlength{\rcollength}{1.95in}%
%
\begin{tabular}[t]{@{}p{\textwidth-\rcollength}p{\rcollength}}
(716) 517--7671 & 2116 College Ave. \\
\email{hewner@rose-hulman.edu} & Terre Haute, IN 47803 \\
\href{http://hewner.com}{http://hewner.com} & \\
\end{tabular}

\section{Objective}

A summer programming position where I can produce some good business value plus learn some things to take back to my students in the fall.
% \begin{innerlist}
% \item More information and auxiliary documents can be found at\\\url{http://www.tedpavlic.com/jobsearch/}
% \end{innerlist}

% \section{Interests}
%
%Student conceptions of the field of Computer Science, affect and identity in education, Computer Science education, Educational Technology, Software Engineering
% Computer Science Education and Computer Science Education Research

\section{Education}
%
\href{http://www.gatech.edu/}{\textbf{Georgia Tech}},
Atlanta, Georgia
\begin{outerlist}

\item[] Ph.D.,
        \href{http://www.ic.gatech.edu/future/phdhcc}
             {Human--Centered Computing},
             December 2012
        \begin{innerlist}
        \item Area of Study: Computer Science Education
        \item Dissertation Topic: \emph{Student Conceptions about the Field of Computer Science}
        \end{innerlist}
\end{outerlist}

\blankline

\href{http://www.uiuc.edu/}{\textbf{University of Illinois at Urbana--Champaign}},
Urbana, Illinois
\begin{outerlist}

\item[] M.S.,
        \href{http://cs.illinois.edu/}
             {Computer Science}, May 2003
        \begin{innerlist}
        \item Area of Study: Software Engineering, Object--Oriented Programming
        \item Thesis Topic: \emph{Implementing the Tagged Integer Optimization on the .NET Virtual Machine}
        \item Adviser:
              \href{http://st-www.cs.illinois.edu/users/johnson/}
                   {Professor Ralph Johnson}
        \end{innerlist}

\item[] B.S.,
        \href{http://cs.illinois.edu/}
             {Computer Science}, December 2001

\end{outerlist}

% \section{Academic Appointments}
% %
% \textbf{Postdoctoral Researcher} \hfill {September 2010 to present}
% \begin{innerlist}

% \item[] \href{http://www.cse.ohio-state.edu/}{Department of Computer Science and Engineering},\\
%         \href{http://www.osu.edu/}{The Ohio State University}
% \begin{innerlist}
% \item \href{http://www.nfs.gov/}{National Science Foundation} Cyber-Physical Systems (ENG, \href{http://www.nsf.gov/div/index.jsp?div=eccs}{ECCS})
% \begin{innerlist}
% \item Autonomous Driving in Mixed-Traffic Urban Environments (\href{http://www.nsf.gov/awardsearch/showAward.do?AwardNumber=0931669}{\#0931669})
% \item Automatic verification of hybrid systems
% \end{innerlist}
% \end{innerlist}

% \end{innerlist}

\section{Industry Experience}
%
\href{http://www.rhventures.org/}{\textbf{Rose--Hulman Ventures}},
Terre Haute, IN
\begin{outerlist}
\item[] \textit{Tech Lead}%
    \hfill \textbf{May 2016--July 2016}
    \begin{innerlist}
    \item Led 2 small programming teams on programming projects for external clients (google glass project, computer vision project) 
    \item Experimental mixed summer internship/instruction partnership with Rose-Hulman Ventures 
    \end{innerlist}
\end{outerlist}
\blankline

\href{http://www.indigobio.com/}{\textbf{Indigo Bioautomation}},
Indianapolis, IN
\begin{outerlist}
\item[] \textit{Programmer}%
    \hfill \textbf{June 2015--August 2015}
    \begin{innerlist}
    \item Wrote Ruby, Java code for mass spectrometer analysis toolchain
    \end{innerlist}
\end{outerlist}
\blankline

\href{http://www.groupon.com/}{\textbf{Groupon}},
San Francisco, CA
\begin{outerlist}
\item[] \textit{Programmer}%
    \hfill \textbf{June 2013--August 2013}
    \begin{innerlist}
    \item Wrote Objective-C (Ipad client--side) and python (django server--side) for Breadcrumb point--of--sale app
    \end{innerlist}
\end{outerlist}
\blankline

\href{http://www.zipperint.com/}{\textbf{Zipper Interactive}},
Seattle, Wahington
\begin{outerlist}
\item[] \textit{Video Game Programmer}%
    \hfill \textbf{May 2009--August 2009}
    \begin{innerlist}
    \item Programmed C++ for two Playstation 3 first person shooter titles
    \item Interviewed developers about what they for in a programmer hire
    \end{innerlist}
\end{outerlist}
\blankline


\href{http://amazon.com}{\textbf{Amazon.com}},
Seattle, Wahington
\begin{outerlist}
\item[] \textit{Software Engineer}%
    \hfill \textbf{June 2003--June 2006, January 2007--July 2007}
    \begin{innerlist}
    \item Technical Lead for a 7 person team, coded many projects in C++ and Perl
    \item Promoted after 1.5 years to SDE II
    \item Developed ``Ninja Coder'' programming riddle project to attract job candidates
    \item Interviewed 100+ developer candidates
    \end{innerlist}
\end{outerlist}
\blankline


\section{Teaching Experience}

\href{http://www.rose-hulman.edu}{\textbf{Rose--Hulman Institute of Technology}},
Terre Haute, Indiana
\begin{outerlist}
\item[] \textit{Assistant Professor}%
    \hfill \textbf{Spring 2013 -- Present}\\
    \begin{innerlist}
    \item CSSE 403: Programming Language Paradigms (senior level course) 
    \item CSSE 220: Intro to Object--Oriented Programming (freshman level course)
    \item CSSE 372: Software Project Management (junior level course)
    \item CSSE 375: Software Construction and Evolution (junior level course)
    \item CSSE 376: Software Quality Assurance (sophomore level course)
    \end{innerlist}

\end{outerlist}

\blankline

\href{http://www.duke.edu}{\textbf{Duke University}},
Durham, North Carolina
\begin{outerlist}
\item[] \textit{Visiting Instructor}%
    \hfill \textbf{Fall 2011 -- Spring 2012}\\
    \begin{innerlist}
    \item CompSci 100: Data Structures (undergraduate course)
    \item CompSci 108: Software Engineering (undergraduate course)
    \item CompSci 149S: Problem Solving Seminar (graduate/undergraduate course)
    \end{innerlist}
\end{outerlist}


\section{References}
%
\href{https://www.linkedin.com/profile/view?id=5565225&authType=NAME_SEARCH&authToken=4hwD&locale=en_US&srchid=56056361421849587692&srchindex=1&srchtotal=188&trk=vsrp_people_res_name&trkInfo=VSRPsearchId%3A56056361421849587692%2CVSRPtargetId%3A5565225%2CVSRPcmpt%3Aprimary}{\textbf{Benjamin Bernard}}
\\ Email: \email{ben@benjaminbernard.com}; Phone: 206--579--4354
\\ CTO Fieldbook (previous team lead Groupon.com)


\blankline


\href{http://www.indigobio.com/tim-lancaster/}{\textbf{Tim Lancaster}}
\\ Email: \email{tlancaster@indigobio.com}; Phone: (317) 605-0841
\\ Vice President of Product Development, Indigo BioAutomation

\blankline

\href{http://www.rose-hulman.edu/academics/academic-departments/computer-science-software-engineering/faculty-staff.aspx}{\textbf{J.P. Mellor}}
\\ Email: \email{mellor@rose-hulman.edu}; Phone: 812--877--8085
\\ Head of Dept. of Computer Science and Software Engineering, Rose--Hulman 


\end{document}

%%%%%%%%%%%%%%%%%%%%%%%%%% End CV Document %%%%%%%%%%%%%%%%%%%%%%%%%%%%%

%----------------------------------------------------------------------%
% The following is copyright and licensing information for
% redistribution of this LaTeX source code; it also includes a liability
% statement. If this source code is not being redistributed to others,
% it may be omitted. It has no effect on the function of the above code.
%----------------------------------------------------------------------%
% Copyright (c) 2007, 2008, 2009, 2010, 2011 by Theodore P. Pavlic
%
% Unless otherwise expressly stated, this work is licensed under the
% Creative Commons Attribution-Noncommercial 3.0 United States License. To
% view a copy of this license, visit
% http://creativecommons.org/licenses/by-nc/3.0/us/ or send a letter to
% Creative Commons, 171 Second Street, Suite 300, San Francisco,
% California, 94105, USA.
%
% THE SOFTWARE IS PROVIDED "AS IS", WITHOUT WARRANTY OF ANY KIND, EXPRESS
% OR IMPLIED, INCLUDING BUT NOT LIMITED TO THE WARRANTIES OF
% MERCHANTABILITY, FITNESS FOR A PARTICULAR PURPOSE AND NONINFRINGEMENT.
% IN NO EVENT SHALL THE AUTHORS OR COPYRIGHT HOLDERS BE LIABLE FOR ANY
% CLAIM, DAMAGES OR OTHER LIABILITY, WHETHER IN AN ACTION OF CONTRACT,
% TORT OR OTHERWISE, ARISING FROM, OUT OF OR IN CONNECTION WITH THE
% SOFTWARE OR THE USE OR OTHER DEALINGS IN THE SOFTWARE.
%----------------------------------------------------------------------%
