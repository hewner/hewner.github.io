
% NOTE:
% This requires the package texlive-latex-extra

%%%%%%%%%%%%%%%%%%%%%%%%%%%%%%%%%%%%%%%%%%%%%%%%%%%%%%%%%%%%%%%%%%%%%%%%
%%%%%%%%%%%%%%%%%%%%%% Simple LaTeX CV Template %%%%%%%%%%%%%%%%%%%%%%%%
%%%%%%%%%%%%%%%%%%%%%%%%%%%%%%%%%%%%%%%%%%%%%%%%%%%%%%%%%%%%%%%%%%%%%%%%

%%%%%%%%%%%%%%%%%%%%%%%%%%%%%%%%%%%%%%%%%%%%%%%%%%%%%%%%%%%%%%%%%%%%%%%%
%% NOTE: If you find that it says                                     %%
%%                                                                    %%
%%                           1 of ??                                  %%
%%                                                                    %%
%% at the bottom of your first page, this means that the AUX file     %%
%% was not available when you ran LaTeX on this source. Simply RERUN  %%
%% LaTeX to get the ``??'' replaced with the number of the last page  %%
%% of the document. The AUX file will be generated on the first run   %%
%% of LaTeX and used on the second run to fill in all of the          %%
%% references.                                                        %%
%%%%%%%%%%%%%%%%%%%%%%%%%%%%%%%%%%%%%%%%%%%%%%%%%%%%%%%%%%%%%%%%%%%%%%%%

%%%%%%%%%%%%%%%%%%%%%%%%%%%% Document Setup %%%%%%%%%%%%%%%%%%%%%%%%%%%%

% Don't like 10pt? Try 11pt or 12pt
\documentclass[10pt]{article}

% This is a helpful package that puts math inside length specifications
\usepackage{calc}


% Simpler bibsection for CV sections
% (thanks to natbib for inspiration)
\makeatletter
\newlength{\bibhang}
\setlength{\bibhang}{1em}
\newlength{\bibsep}
 {\@listi \global\bibsep\itemsep \global\advance\bibsep by\parsep}
\newenvironment{bibsection}%
        {\vspace{-\baselineskip}\begin{list}{}{%
       \setlength{\leftmargin}{\bibhang}%
       \setlength{\itemindent}{-\leftmargin}%
       \setlength{\itemsep}{\bibsep}%
       \setlength{\parsep}{\z@}%
        \setlength{\partopsep}{0pt}%
        \setlength{\topsep}{0pt}}}
        {\end{list}\vspace{-.6\baselineskip}}
\makeatother

% Layout: Puts the section titles on left side of page
\reversemarginpar

%
%         PAPER SIZE, PAGE NUMBER, AND DOCUMENT LAYOUT NOTES:
%
% The next \usepackage line changes the layout for CV style section
% headings as marginal notes. It also sets up the paper size as either
% letter or A4. By default, letter was used. If A4 paper is desired,
% comment out the letterpaper lines and uncomment the a4paper lines.
%
% As you can see, the margin widths and section title widths can be
% easily adjusted.
%
% ALSO: Notice that the includefoot option can be commented OUT in order
% to put the PAGE NUMBER *IN* the bottom margin. This will make the
% effective text area larger.
%
% IF YOU WISH TO REMOVE THE ``of LASTPAGE'' next to each page number,
% see the note about the +LP and -LP lines below. Comment out the +LP
% and uncomment the -LP.
%
% IF YOU WISH TO REMOVE PAGE NUMBERS, be sure that the includefoot line
% is uncommented and ALSO uncomment the \pagestyle{empty} a few lines
% below.
%

%% Use these lines for letter-sized paper
\usepackage[paper=letterpaper,
            %includefoot, % Uncomment to put page number above margin
            marginparwidth=1.2in,     % Length of section titles
            marginparsep=.05in,       % Space between titles and text
            margin=1in,               % 1 inch margins
            includemp]{geometry}

%% Use these lines for A4-sized paper
%\usepackage[paper=a4paper,
%            %includefoot, % Uncomment to put page number above margin
%            marginparwidth=30.5mm,    % Length of section titles
%            marginparsep=1.5mm,       % Space between titles and text
%            margin=25mm,              % 25mm margins
%            includemp]{geometry}

%% More layout: Get rid of indenting throughout entire document
\setlength{\parindent}{0in}

%% This gives us fun enumeration environments. compactitem will be nice.
\usepackage{paralist}

%% Reference the last page in the page number
%
% NOTE: comment the +LP line and uncomment the -LP line to have page
%       numbers without the ``of ##'' last page reference)
%
% NOTE: uncomment the \pagestyle{empty} line to get rid of all page
%       numbers (make sure includefoot is commented out above)
%
\usepackage{fancyhdr,lastpage}
\pagestyle{fancy}
%\pagestyle{empty}      % Uncomment this to get rid of page numbers
\fancyhf{}\renewcommand{\headrulewidth}{0pt}
\fancyfootoffset{\marginparsep+\marginparwidth}
\newlength{\footpageshift}
\setlength{\footpageshift}
          {0.5\textwidth+0.5\marginparsep+0.5\marginparwidth-2in}
\lfoot{\hspace{\footpageshift}%
       \parbox{4in}{\, \hfill %
                    \arabic{page} of \protect\pageref*{LastPage} % +LP
%                    \arabic{page}                               % -LP
                    \hfill \,}}

% Finally, give us PDF bookmarks
\usepackage{color,hyperref}
\definecolor{darkblue}{rgb}{0.0,0.0,0.3}
\hypersetup{colorlinks,breaklinks,
            linkcolor=darkblue,urlcolor=darkblue,
            anchorcolor=darkblue,citecolor=darkblue}

%%%%%%%%%%%%%%%%%%%%%%%% End Document Setup %%%%%%%%%%%%%%%%%%%%%%%%%%%%


%%%%%%%%%%%%%%%%%%%%%%%%%%% Helper Commands %%%%%%%%%%%%%%%%%%%%%%%%%%%%

% The title (name) with a horizontal rule under it
%
% Usage: \makeheading{name}
%
% Place at top of document. It should be the first thing.
\newcommand{\makeheading}[1]%
        {\hspace*{-\marginparsep minus \marginparwidth}%
         \begin{minipage}[t]{\textwidth+\marginparwidth+\marginparsep}%
                {\large \bfseries #1}\\[-0.15\baselineskip]%
                 \rule{\columnwidth}{1pt}%
         \end{minipage}}

% The section headings
%
% Usage: \section{section name}
%
% Follow this section IMMEDIATELY with the first line of the section
% text. Do not put whitespace in between. That is, do this:
%
%       \section{My Information}
%       Here is my information.
%
% and NOT this:
%
%       \section{My Information}
%
%       Here is my information.
%
% Otherwise the top of the section header will not line up with the top
% of the section. Of course, using a single comment character (%) on
% empty lines allows for the function of the first example with the
% readability of the second example.
\renewcommand{\section}[2]%
        {\pagebreak[3]\vspace{1.3\baselineskip}%
         \phantomsection\addcontentsline{toc}{section}{#1}%
         \hspace{0in}%
         \marginpar{
         \raggedright \scshape #1}#2}

% An itemize-style list with lots of space between items
\newenvironment{outerlist}[1][\enskip\textbullet]%
        {\begin{itemize}[#1]}{\end{itemize}%
         \vspace{-.6\baselineskip}}

% An environment IDENTICAL to outerlist that has better pre-list spacing
% when used as the first thing in a \section
\newenvironment{lonelist}[1][\enskip\textbullet]%
        {\vspace{-\baselineskip}\begin{list}{#1}{%
        \setlength{\partopsep}{0pt}%
        \setlength{\topsep}{0pt}}}
        {\end{list}\vspace{-.6\baselineskip}}

% An itemize-style list with little space between items
\newenvironment{innerlist}[1][\enskip\textbullet]%
        {\begin{compactitem}[#1]}{\end{compactitem}}

% An environment IDENTICAL to innerlist that has better pre-list spacing
% when used as the first thing in a \section
\newenvironment{loneinnerlist}[1][\enskip\textbullet]%
        {\vspace{-\baselineskip}\begin{compactitem}[#1]}
        {\end{compactitem}\vspace{-.6\baselineskip}}

% To add some paragraph space between lines.
% This also tells LaTeX to preferably break a page on one of these gaps
% if there is a needed pagebreak nearby.
\newcommand{\blankline}{\quad\pagebreak[2]}

% Uses hyperref to link DOI
\newcommand\doilink[1]{\href{http://dx.doi.org/#1}{#1}}
\newcommand\doi[1]{doi:\doilink{#1}}

% For \url{SOME_URL}, links SOME_URL to the url SOME_URL
\providecommand*\url[1]{\href{#1}{#1}}
% Same as above, but pretty-prints SOME_URL in teletype fixed-width font
\renewcommand*\url[1]{\href{#1}{\texttt{#1}}}

% For \email{ADDRESS}, links ADDRESS to the url mailto:ADDRESS
\providecommand*\email[1]{\href{mailto:#1}{#1}}
% Same as above, but pretty-prints ADDRESS in teletype fixed-width font
%\renewcommand*\email[1]{\href{mailto:#1}{\texttt{#1}}}

\newcommand{\respub}[4]{
  \item #2.  \emph{#1.}  #3.  #4.
}

%%%%%%%%%%%%%%%%%%%%%%%% End Helper Commands %%%%%%%%%%%%%%%%%%%%%%%%%%%

%%%%%%%%%%%%%%%%%%%%%%%%% Begin CV Document %%%%%%%%%%%%%%%%%%%%%%%%%%%%

\begin{document}
\makeheading{Michael Hewner}

\section{Contact Information}
%
% NOTE: Mind where the & separators and \\ breaks are in the following
%       table.
%
% ALSO: \rcollength is the width of the right column of the table
%       (adjust it to your liking; default is 1.85in).
%
\newlength{\rcollength}\setlength{\rcollength}{1.95in}%
%
\begin{tabular}[t]{@{}p{\textwidth-\rcollength}p{\rcollength}}
(716) 517--7671 & 2116 College Ave. \\
\email{hewner@rose-hulman.edu} & Terre Haute, IN 47803 \\
\href{http://hewner.com}{http://hewner.com} & \\
\end{tabular}

%\section{Objective}
%
%Placement in a Computer Science teaching position that will also allow me to continue my research into undergraduate Computer Science education.
% \begin{innerlist}
% \item More information and auxiliary documents can be found at\\\url{http://www.tedpavlic.com/jobsearch/}
% \end{innerlist}

\section{Interests}
%
%Student conceptions of the field of Computer Science, affect and identity in education, Computer Science education, Educational Technology, Software Engineering
Computer Science Education and Computer Science Education Research

\section{Education}
%
\href{http://www.gatech.edu/}{\textbf{Georgia Tech}},
Atlanta, Georgia
\begin{outerlist}

\item[] Ph.D.,
        \href{http://www.ic.gatech.edu/future/phdhcc}
             {Human--Centered Computing},
             December 2012
        \begin{innerlist}
        \item Area of Study: Computer Science Education
        \item Dissertation Topic: \emph{Student Conceptions about the Field of Computer Science}
        \item Adviser:
              \href{http://www.cc.gatech.edu/~mark.guzdial/}
                   {Professor Mark Guzdial}
        \item \href{http://www.techtoteach.gatech.edu/highered/teaching.htm}{Higher Education Teaching Certificate Level A}
        \end{innerlist}
\end{outerlist}

\blankline

\href{http://www.uiuc.edu/}{\textbf{University of Illinois at Urbana--Champaign}},
Urbana, Illinois
\begin{outerlist}

\item[] M.S.,
        \href{http://cs.illinois.edu/}
             {Computer Science}, May 2003
        \begin{innerlist}
        \item Area of Study: Software Engineering, Object--Oriented Programming
        \item Thesis Topic: \emph{Implementing the Tagged Integer Optimization on the .NET Virtual Machine}
        \item Adviser:
              \href{http://st-www.cs.illinois.edu/users/johnson/}
                   {Professor Ralph Johnson}
        \end{innerlist}

\item[] B.S.,
        \href{http://cs.illinois.edu/}
             {Computer Science}, December 2001

\end{outerlist}

% \section{Academic Appointments}
% %
% \textbf{Postdoctoral Researcher} \hfill {September 2010 to present}
% \begin{innerlist}

% \item[] \href{http://www.cse.ohio-state.edu/}{Department of Computer Science and Engineering},\\
%         \href{http://www.osu.edu/}{The Ohio State University}
% \begin{innerlist}
% \item \href{http://www.nfs.gov/}{National Science Foundation} Cyber-Physical Systems (ENG, \href{http://www.nsf.gov/div/index.jsp?div=eccs}{ECCS})
% \begin{innerlist}
% \item Autonomous Driving in Mixed-Traffic Urban Environments (\href{http://www.nsf.gov/awardsearch/showAward.do?AwardNumber=0931669}{\#0931669})
% \item Automatic verification of hybrid systems
% \end{innerlist}
% \end{innerlist}

% \end{innerlist}

\section{Instructor of Record}

\href{http://www.rose-hulman.edu}{\textbf{Rose--Hulman Institute of Technology}},
Terre Haute, Indiana
\begin{outerlist}
\item[] \textit{Assistant Professor}%
    \hfill \textbf{Spring 2013 -- Present}\\

    CSSE 220: Intro to Object--Oriented Programming (freshman level course)
    \begin{innerlist}
    \item Topics: java, object oriented design, basic algorithms and data structures
    \item Mixed instruction, in class programming, and projects
    \item Taught 9 times % last was Fall 17 
%    \item Winter 2014, 60 students
    \end{innerlist}

    CSSE 290: Advanced GIT (1 credit elective course)
    \begin{innerlist}
    \item Topics: git internals, merging/rebasing, branch design
    \item Useful course that tends to get a lot of student interest
    \item Taught 3 times % last was Winter 16 
    \end{innerlist}

    CSSE 333: Databases 
    \begin{innerlist}
    \item Topics: SQL, DB design, web-integration
    \item Experimental version for freshman, part of a mixed summer internship/instruction partnership with Rose-Hulman Ventures
    \item Taught Summer 2016
    \end{innerlist}

    CSSE 372: Software Project Management (junior level course)
    \begin{innerlist}
    \item Topics: software processes, estimation, risk management, planning
    \item Discussion--oriented course
    \item Taught 2 times % last was W13
    \end{innerlist}

    CSSE 375: Software Construction and Evolution (junior level course)
    \begin{innerlist}
    \item Topics: refactoring, advanced OO--design, large scale systems
    \item Lecture--based course with programming assignments and year long project
    \item Taught 2 times
    \end{innerlist}

    CSSE 403: Programming Language Paradigms (senior level course)
    \begin{innerlist}
    \item Survey of interesting languages: Prolog, Erlang, Elm
    \item Project--oriented course, but also regular lectures
    \item Taught 3 times
    \end{innerlist}

    Also Taught
    \begin{innerlist}
    \item CSSE290: Cyberdefense Competition
    \item CSSE332: Software Quality Assurance
    \item CSSE497, CSSE498: Senior Project
    \item CSSE376: Software Quality Assurance
    \end{innerlist}


\end{outerlist}

\blankline

\href{http://www.duke.edu}{\textbf{Duke University}},
Durham, North Carolina
\begin{outerlist}
\item[] \textit{Visiting Instructor}%
    \hfill \textbf{Fall 2011 -- Spring 2012}\\
    CompSci 100: Data Structures (undergraduate course)
    \begin{innerlist}
    \item Topics: algorithm design, objects, recursion, linked--lists, trees
    \item Lecture--based course with programming assignments and exams
    \item Taught 150+ students with another instructor in Fall, taught alone in Spring
    \item Developed lectures, wrote exams
    \end{innerlist}

    CompSci 108: Software Engineering (undergraduate course)
    \begin{innerlist}
    \item Topics: object--oriented design, programming large systems
    \item Project--oriented course, but also regular lectures
    \item Taught 40+ students, with another instructor in Fall, taught alone in Spring
    \item Developed lectures, developed projects and grading criteria
    \end{innerlist}

    % CompSci 149S: Problem Solving Seminar (graduate/undergraduate course)
    % \begin{innerlist}
    % \item Topics: programming for programming competitions, algorithms
    % \item Taught 10 students
    % \item Used various styles including mini--lectures, discussions, in--class scrimmages
    % \item Team won 1st Place in ACM ICPC Regionals
    % \end{innerlist}
\end{outerlist}

\blankline

\href{http://www.uw.edu}{\textbf{University of Washington}},
Seattle, Washington
\begin{outerlist}
\item[] \textit{Visiting Instructor}%
    \hfill \textbf{Summer 2008}\\
    CSE143: Computer Programming II (undergraduate course)
    \begin{innerlist}
    \item Topics: algorithm design, objects, recursion, linked--lists, trees
    \item Taught 80+ students
    \item Developed lectures, exams, managed TAs
    \end{innerlist}
\end{outerlist}

\section{Other Teaching Experience}

\href{http://www.iitb.ac.in/}{\textbf{Indian Institute of Technology Bombay}},
Mumbai, India

\begin{outerlist}

\item[] \textit{Visiting Assistant Professor}%
    \hfill \textbf{Summer 2014}\\
    Qualitative Methods in Engineering Education (graduate seminar)
    \begin{innerlist}
    \item Topics: interviewing, grounded theory, content analysis
    \item Also advised students on research topics/approaches
    \item 20 students
    \end{innerlist}
\end{outerlist}


\href{http://www.valdosta.edu/ghp}{\textbf{Governor's Honors Program}},
Valdosta, Georgia

A competitive 4--week summer program for high school juniors sponsored by the state of Georgia
\begin{outerlist}

\item[] \textit{Instructor}%
    \hfill \textbf{Summer 2011, Summer 2012}\\
    Introductory Delphi Programming (high school course)
    \begin{innerlist}
    \item Topics: variables, functions, GUIs, Monte Carlo simulations, complex math
    \item 20 students
    \end{innerlist}

    Intro to Automata Theory (high school course)
    \begin{innerlist}
    \item Topics: different types of automata, incomputability, Turing--Church Thesis
    \item 15 students
    \end{innerlist}

    Fractals (high school course)
    \begin{innerlist}
    \item Topics: Iterated function systems, fractal dimension, chaos
    \item 15 students
    \end{innerlist}

% \item[] \textit{Math Lab Tech and Volunteer Instructor}%
%     \hfill \textbf{Summer 2010}\\
%     Introductory Delphi Programming (high school course)
%     \begin{innerlist}
%     \item Topics: variables, functions, GUIs, Monte Carlo simulations, complex math
%     \item Taught 40 students (2 sections) per day with instructor supervision
%     \end{innerlist}
\end{outerlist}

% \blankline

% \href{http://www.gatech.edu}{\textbf{Georgia Tech}},
% Atlanta, Georgia
% \begin{outerlist}

% \item[] \textit{Student Mentor (unofficial Teaching Assistant)}%
%     \hfill \textbf{Fall 2010}\\
%     CS 6452: Rapid Prototyping (graduate course)
%     \begin{innerlist}
%     \item Topics: Jython, GUI frameworks, networking, OO--design, databases
%     \item 15 students
%     \item Taught 2 lectures: Python Fundamentals and Databases
%     \item Held regular office hours, responded to student emails
%     \end{innerlist}
% \item[] \textit{Teaching Practicum}%
%     \hfill \textbf{Fall 2009}\\
%     CS 2110: Computer Organization and Programming (sophomore course)
%     \begin{innerlist}
%     \item Topics: processor architecture, assembly language, C
%     \item 80+ students
%     \item Taught guest lectures on Logic Gates, Memory Mapped IO, and The Stack/Malloc
%     \item Observed classes, TA recitations, and met weekly with teacher and TAs to discuss teaching
%     \end{innerlist}
% \end{outerlist}

% \blankline


% \href{http://www.uw.edu}{\textbf{University of Illinois}},
% Urbana, Illinois
% \begin{outerlist}
% \item[] \textit{Teaching Assistant}%
%     \hfill \textbf{Spring 2002--Spring 2003}\\
%     Software Engineering I and II (mixed graduate/undergraduate course)
%     \begin{innerlist}
%     \item Topics: software processes, UML, object--oriented design, project management, software tools
%     \item TA for 80+ students
%     \item Held regular office hours, managed student project work
%     \item Developed and graded homeworks and exams
%     \item Led other TAs
%     \end{innerlist}
% \end{outerlist}

% \section{Guest Lectures}
% \begin{innerlist}
% \item Technology and Society Class: Privacy and Anonymity (Novemeber 2011)
% \item Introductory Data Structures Class: Simulations (April 2011)
% \item CS TA Training Class: Educational Objectives (April 2010)
% \item CS TA Training Class: Active Learning (April 2010)
% \item Educational Technology Class: Identity and CS Education (November 2009)
% \item Educational Technology Class: Resnick and Distributed Thinking (February 2009)
% \end{innerlist}

% \section{High School Outreach}
%  \textit{Mini--courses}
%   \begin{innerlist}
%     \item High School Mentor Training: Three Lectures on the Subfields of CS and Student Goals (Spring 2010)
%   \end{innerlist}
% \textit{Class Presentations}%
%   \begin{innerlist}
%     \item High School Presentation: Careers in Video Game Programming (August 2010 and February 2011)
%     \item High School Presentation: Subfields of CS (Fall 2008)
%   \end{innerlist}
% \textit{Student Mentoring}%
%   \begin{innerlist}
%   \item Mentored a high school team doing a project entitled ``Using an Intermediate Neural Network to Optimize Parameters in Backpropagation Neural Networks'' for the Siemens Science Fair Competition (Fall 2011)
%   \item Tutored math at Rainier Beach High School (Fall 2006)
%   \item Mentored high school students as part of \href{http://www.communityforyouth.org/}{Community for Youth} program (Fall 2004---Fall 2006)
%   \end{innerlist}


\section{Publications}
\begin{bibsection}

\respub{When Everyone Knows CS is the Best Major: Decisions about CS in an Indian context}{M. Hewner and S. Mishra}{presented at Twelfth International Computing Education Research Workshop (ICER 2016)}{Melbourne, Australia, September 8-12, 2016}

\respub{How Undergraduates Make Course Choices}{M. Hewner}{presented at Tenth International Computing Education Research Workshop (ICER 2014)}{Glasgow UK, August 11-14, 2014}

\respub{Undergraduate Conceptions of the Field of Computer Science}{M. Hewner}{presented at Ninth International Computing Education Research Workshop (ICER 2013)}{San Diego, CA USA, August 12-14, 2013}

\respub{How CS majors select a specialization}{M. Hewner and M. Guzdial}{presented at Seventh International Computing Education Research Workshop (ICER 2011)}{Providence, RI USA, August 8-9, 2011}

\respub{What Game Developers Look for in a New Graduate: Interviews and Surveys at One Game Company}{M. Hewner and M. Guzdial}{presented at ACM Technical Symposium on Computer Science Education (SIGCSE 2010)}{Milwaukee, WI USA, March 10-13, 2010}

\respub{`Georgia computes!': improving the computing education pipeline}{A. Bruckman, M. Biggers, B. Ericson, T. McKiln, J. Dimond, B. DiSalvo, M. Hewner, L. Ni, S. Yardi}{presented at ACM Technical Symposium on Computer Science Education (SIGCSE 2009)}{Chattanooga, TN USA, March 4-7, 2009}

\respub{Understanding Computing Stereotypes with Self-Categorization Theory}{M. Hewner and M. Knobelsdorf}{presented at Koli Calling International Conference on Computer Science Education (Koli Calling 2008)}{Koli National Park, Finland, November 13 - 16, 2008}

\respub{Attitudes about Computing in Postsecondary Graduates}{M. Hewner and M. Guzdial}{presented at Fourth International Computing Education Research Workshop (ICER 2008)}{Sydney, Australia, September 6-7 2008}

\end{bibsection}

\section{Industry Experience}
%
\href{http://www.rhventures.org/}{\textbf{Rose--Hulman Ventures}},
Terre Haute, IN
\begin{outerlist}
\item[] \textit{Tech Lead}%
    \hfill \textbf{May 2016--July 2016}
    \begin{innerlist}
    \item Manager and technical adviser for two teams of freshman CS
      students doing contract software development
    \item Experimental version for freshman, part of a mixed summer internship/instruction partnership with Rose-Hulman Ventures 
    \end{innerlist}
\end{outerlist}
\blankline


\href{http://www.indigobio.com/}{\textbf{Indigo Bioautomation}},
Indianapolis, IN
\begin{outerlist}
\item[] \textit{Programmer}%
    \hfill \textbf{June 2015--August 2015}
    \begin{innerlist}
    \item Wrote Ruby, Java code for mass spectrometer analysis toolchain
    \end{innerlist}
\end{outerlist}
\blankline

\href{http://www.groupon.com/}{\textbf{Groupon}},
San Francisco, CA
\begin{outerlist}
\item[] \textit{Programmer}%
    \hfill \textbf{June 2013--August 2013}
    \begin{innerlist}
    \item Wrote Objective-C (Ipad client--side) and python (django server--side) for Breadcrumb point--of--sale app
    \end{innerlist}
\end{outerlist}
\blankline

\href{http://www.zipperint.com/}{\textbf{Zipper Interactive}},
Seattle, Wahington
\begin{outerlist}
\item[] \textit{Video Game Programmer}%
    \hfill \textbf{May 2009--August 2009}
    \begin{innerlist}
    \item Programmed C++ for two Playstation 3 first person shooter titles
    \item Interviewed developers about what they for in a programmer hire
    \end{innerlist}
\end{outerlist}
\blankline


\href{http://amazon.com}{\textbf{Amazon.com}},
Seattle, Wahington
\begin{outerlist}
\item[] \textit{Software Engineer}%
    \hfill \textbf{June 2003--June 2006, January 2007--July 2007}
    \begin{innerlist}
    \item Technical Lead for a 7 person team, coded many projects in C++ and Perl
    \item Promoted after 1.5 years to SDE II
    \item Developed ``Ninja Coder'' programming riddle project to attract job candidates
    \item Interviewed 100+ developer candidates
    \end{innerlist}
\end{outerlist}
\blankline

% \href{http://progressive.com}{\textbf{Progressive Insurance}},
% Cleveland, Ohio
% \begin{outerlist}
% \item[] \textit{Developer Intern}%
%     \hfill \textbf{Summer 2002, Summer 2001}
%     \begin{innerlist}
%     \item Programmed Smalltalk for insurance rate setting system
%     \item Programmed Visual Basic for Progressive website
%     \end{innerlist}
% \end{outerlist}
% \blankline

% \href{http://www.ncsa.illinois.edu/}{\textbf{National Center for Supercomputing Applications}},
% Urbana, Illinois
% \begin{outerlist}
% \item[] \textit{Student Programmer}%
%     \hfill \textbf{December 1999--September 2000}
%     \begin{innerlist}
%     \item Worked on Java system for predicting molecular structure
%     \item Programmed system for atom categorization
%     \end{innerlist}
% \end{outerlist}

\section{Service}

\begin{innerlist}
% \item Advisor for Rose--Hulman Boardgames Club
\item Session Chair for SIGCSE 2012
\item Coach of the Rose--Hulman CCDC Team, Security Club (Spring 2013 -- Fall 2015)
\item Coach of Duke Programming Competition Team (Fall 2011 -- Spring 2012)
\item Student representative on HCC Ph.D. Procedure Review Committee (Spring 2011)
\item Paper reviewer for ICER, SIGCSE, and TOCE
\end{innerlist}


\section{References}

\href{http://www.rose-hulman.edu/academics/academic-departments/computer-science-software-engineering/faculty-staff.aspx}{\textbf{J.P. Mellor}}
\\ Email: \email{mellor@rose-hulman.edu}; Phone: 812--877--8085
\begin{innerlist}
    \item Head of Dept. of Computer Science and Software Engineering, Rose--Hulman 
    \item[$\diamond$] \emph{Current boss}
\end{innerlist}

\blankline

%
\href{http://www.cc.gatech.edu/~mark.guzdial/}{\textbf{Mark Guzdial}}
\\ Email: \email{guzdial@cc.gatech.edu}; Phone: 404--894--5618
\begin{innerlist}
    \item Professor, Georgia Tech
    \item[$\diamond$] \emph{Dissertation adviser}
\end{innerlist}

\blankline

\href{http://www.cs.duke.edu/~ola/}{\textbf{Owen Astrachan}}
\\ Email: \email{ola@cs.duke.edu}; Phone: (919) 660-6522
\begin{innerlist}
    \item Professor of the Practice, Duke University
    \item[$\diamond$] \emph{Co--Instructor in Data Structures Course}
\end{innerlist}

\blankline

\href{http://www.cs.kent.ac.uk/people/staff/saf/}{\textbf{Sally Fincher}}
\\ Email: \email{s.a.fincher@kent.ac.uk}; Phone: +44 (0)1227 824061
\begin{innerlist}
    \item Professor, University of Kent
    \item[$\diamond$] \emph{Can speak to my qualifications as a CS Education Researcher}
\end{innerlist}


% \blankline

% \href{http://www.cc.gatech.edu/~keith/}{\textbf{Keith Edwards}}
% \\ Email: \email{keith@cc.gatech.edu}; Phone: 404--385--6783
% \begin{innerlist}
%     \item Professor, Georgia Tech
%     \item[$\diamond$] \emph{Observed my teaching/student interactions in the Rapid Prototyping course, Dissertation committee member}
% \end{innerlist}

% \blankline

% \href{http://st-www.cs.illinois.edu/users/johnson/}{\textbf{Ralph Johnson}}
% \\ Email: \email{johnson@cs.uiuc.edu}; Phone: 217--244--0093
% \begin{innerlist}
%     \item Professor, University of Illinois at Urbana--Champaign
%     \item[$\diamond$] \emph{Masters adviser, supervisor for the Software Engineering I \& II TA position}
% \end{innerlist}

\end{document}

%%%%%%%%%%%%%%%%%%%%%%%%%% End CV Document %%%%%%%%%%%%%%%%%%%%%%%%%%%%%

%----------------------------------------------------------------------%
% The following is copyright and licensing information for
% redistribution of this LaTeX source code; it also includes a liability
% statement. If this source code is not being redistributed to others,
% it may be omitted. It has no effect on the function of the above code.
%----------------------------------------------------------------------%
% Copyright (c) 2007, 2008, 2009, 2010, 2011 by Theodore P. Pavlic
%
% Unless otherwise expressly stated, this work is licensed under the
% Creative Commons Attribution-Noncommercial 3.0 United States License. To
% view a copy of this license, visit
% http://creativecommons.org/licenses/by-nc/3.0/us/ or send a letter to
% Creative Commons, 171 Second Street, Suite 300, San Francisco,
% California, 94105, USA.
%
% THE SOFTWARE IS PROVIDED "AS IS", WITHOUT WARRANTY OF ANY KIND, EXPRESS
% OR IMPLIED, INCLUDING BUT NOT LIMITED TO THE WARRANTIES OF
% MERCHANTABILITY, FITNESS FOR A PARTICULAR PURPOSE AND NONINFRINGEMENT.
% IN NO EVENT SHALL THE AUTHORS OR COPYRIGHT HOLDERS BE LIABLE FOR ANY
% CLAIM, DAMAGES OR OTHER LIABILITY, WHETHER IN AN ACTION OF CONTRACT,
% TORT OR OTHERWISE, ARISING FROM, OUT OF OR IN CONNECTION WITH THE
% SOFTWARE OR THE USE OR OTHER DEALINGS IN THE SOFTWARE.
%----------------------------------------------------------------------%
